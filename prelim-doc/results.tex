\chapter{Formal Semantics of \ISA User-Level ISA}\label{sec:results}
The equivalence checker being parametric on the semantics of source (\ISA) and target (LLVM) languages of decompilation, one of the important steps towards the goal is to define these semantics. In this section, we will discuss our published contribution~\cite{DasguptaAdve:PLDI19} is coming up with the most complete and thoroughly tested formal
semantics of x86-64 to date.


\section{Challenges in Formalizing \ISA}
\label{sec:challenges-in-formalizing-x86}
The \ISA instruction set architecture (ISA) is one of the most complex and widely used ISAs on servers and desktops, and ensuring the correctness of the \ISA binary code is important.
%
Completely formalizing the semantics of \ISA, however, is challenging especially due to the complexity and the sheer number of instructions that are informally specified in approximately 3,000-page standard~\cite{IntelManual}.
% In addition to the sheer number of instructions to be specified,
% % and the ambiguity of the informal standard document to consult,
% the following aspects make it challenging to completely specify the formal semantics of \ISA.

\paragraph{Size and Complexity:}
%
\SC{The \ISA ISA has a large number of instructions, partly because of a large number of complex instructions and partly because it keeps most of the legacy and deprecated instructions ($\sim$ \Xmmx{}+) for the sake of backwards compatibility.}
It consists of \totalIntel{} mnemonics, and each mnemonic admits several variants, depending on the types (i.e., register, memory, or constant) and the size (i.e., the bit-width) of operands.

\paragraph{Inconsistent Instruction Variants:}
%
Some variants have divergent behaviors more than the difference of their type and size. For example, \instr{movsd}, one of the 128-bit SSE instructions, has very different behaviors depending on whether the type of the source operand is register or memory; it clears the higher 64 bits of the target register only when the source type is memory.
Indeed, we revealed bugs in other semantics due to their incorrect generalization of the variants' behavior.

\paragraph{Ambiguous Documentation:}
%
The \ISA reference manual informally explains the instruction behaviors, leaving certain details unspecified or ambiguous, which required us to consult with an actual processor implementation to clarify such details.
%
Completely formalizing the vast number of instructions with carefully identifying all the corner cases from the informal document, thus, is highly non-trivial.


\paragraph{Undefined Behaviors:}
%
The \ISA standard also admits undefined behaviors that are implementation-dependent.
Many instructions (\undefIntel{}\footnote{\label{note1}These numbers are obtained by parsing the official manual ``Volume 2: Instruction Set Reference'' and cross checked with projects~\cite{Stoke2013, Felix} investing similar efforts.} out of \totalIntel{} mnemonics) have undefined behaviors: their output values of the destination register or the \reg{rflags} register are undefined in certain cases.
That is, the processor is free to choose any behavior in the undefined case.


Many existing semantics, however, simply ``define'' the undefined behaviors by
following a specific behavior taken by a processor implementation.
This approach is problematic because they do not capture all possible behaviors of different processor implementations.
Indeed, we found discrepancies between existing semantics in specifying the undefined behaviors, where different semantics are valid only for different groups of processors.
That is, such semantics are not adequate to formally reason about universal properties (e.g., portability) of a program that need to be satisfied for all standard-conforming processors.
%
For example, the parity flag \reg{pf} is undefined in the logical-and-not instruction \instr{andn}, where the processor implementation is allowed to either update the flag value (to 0 or 1), or keep the previous value.
We found, e.g., that Remill~\cite{Remill} updates the flag with 0, whereas Radare~\cite{Radare2} keeps it unmodified.
Identifying and faithfully specifying all the undefined behaviors, thus, are desirable but challenging.

\SC{In our semantics, we faithfully modeled the undefined value as a unique symbol (called \s{undef}) whose value is nondeterministically decided each time within the proper range.}
These nondeterministic values are enough to capture and formally reason about all possible behaviors of the instructions for different processors (and even any future, standard-conforming processor).

\section{Existing Semantics for \ISA}

To date, to the best of our knowledge, despite several explicit attempts~\cite{Heule2016a,Goel:FMCAD14,Goel:ProCoS17} and other related systems~\cite{Leroy:2009,Remill,TSL:TOPLAS13,Hasabnis:ASPLOS16,Hasabnis:FSE16},
no \emph{complete} formal semantics of \ISA exists that can be used for formal reasoning about x86 binary programs.
%
Heule~\etal \cite{Heule2016a} presented a formal semantics of \ISA, but it covers only a fragment ($\sim$\strataPerc{}) of all instructions; as the authors of \cite{Heule2016a} candidly admitted, their synthesis methodology proved insufficient to add the remaining instructions primarily due to limitations of the underlying synthesis engine. 
%
Moreover, their semantics misses certain essential details~\cite{DasguptaAdve:PLDI19}.
% and it has not been demonstrated how to use their semantics to formally reason about the functional correctness of the \ISA binary.\footnote{Although they provide a tool that can extract SMT formulae describing each instruction's behavior from their semantics, we found errors in their tool and thus the generated SMT formulae. Also, most of the SMT solvers are not designed to support rich reasoning principles.}
% natively support all the reasoning principles of the full-fledged proof assistants.}
% for the general purpose formal reasoning.}
%
% it is not clear how to lift their semantics to the full-fledged theorem prover to be able to reason about the full functional correctness of the \ISA binary.
% Although it can provide the semantics in the form of the SMT formulae, the SMT solver 
%
Goel~\etal~\cite{Goel:FMCAD14,Goel:ProCoS17}, on the other hand, specified a formal semantics in the \SC{ACL2} proof assistant~\cite{ACL2:Kaufmann2000}, allowing to reason about functional correctness, but their semantics covers only a small fragment ($\sim$\goelPerc{}) of all user-level instructions.
%\Comment{Section 2 says they had only 191 unique opcodes, which is only about 20\%, and even less if you include system instructions. Which is correct?}
%
There also have been several attempts~\cite{Angr1,BAP:CAV11,Radare2,Hasabnis:FSE16} to \emph{indirectly} describe the \ISA semantics, where they define an intermediate language (IL), specify the IL semantics, and translate \ISA to the IL.
This indirect semantics, however, may not be general enough to be used for different types of formal analyses, since the IL might be designed with specific purposes in mind, not to mention that the translation may miss certain important details of the instruction behaviors.
%
\section{Overview of the Approach}
\label{sec:Approach:Overview}

Briefly, our approach is as follows.
%
We first defined the machine configuration and underlying infrastructure in the \K framework, in order to define, execute and test the \ISA semantics.
%
\SC{To leverage previous work as much as possible, we took the semantic rules of all the instructions supported in Strata~\cite{Heule2016a}, which amounts to about 60\% of the instructions in scope, in the form of SMT formulas.}
%
We corrected, improved or simplified many of the baseline rules.
%
We then translated these SMT formulas from Strata into \K rules using a script, and tested the resulting rules by comparing with the Strata rules using Z3.
%
These steps give us a validated initial set of semantic rules in \K for about 60\% of the target instructions (our ``baseline'' set).

We attempted to extend the stratification approach in Strata to define additional rules automatically, in two ways: (i) augmenting their base set \s{B}, and (ii) constraining the search space manually using knowledge of instruction behaviors.  Both these attempts failed; they worked only for a few instructions, but in general, we found them to be impractical. Specifically, we added $58$ base instructions to the base set, and learned the semantics of $70$ new instructions, which are variants of the added  instructions, in $20$ minutes, but no more even after we kept running for two days. Also, we tried constraining the search space by manually populating it with relevant instructions. The lesson we learned from these experiments is, getting the right set of base instructions or a constrained search space for a complex instruction need an insight about the semantics of that instruction itself. We found that the effort to extract such information from the manual is about the same  as manually defining that instruction.


%\Comment{SANDEEP: Please check this last sentence and fix / improve.}

We then manually added \K rules for the remaining 40\% of the target instructions by systematically translating their description of the Intel manual into \K rules, in some cases cross-referencing against semantics available in Stoke.
%
The outcome was a complete formal specification of user-level \ISA in \K.

We validated this semantics in three ways:
%
First, we use the \K interpreter to execute the semantics of \emph{each} instruction for 7,000+ test inputs (each input is a processor state configuration) and compared the output directly with the hardware behavior for the same instruction.
%
Second, we repeated this experiment using the applicable programs in the \SC{GCC C-torture tests~\cite{CTORTURE}}.  % (\TortureInclude{} out of \TortureTotal{} tests).
%
Third, we compared against the semantics defined in the Stoke project for about 330 instructions that were omitted in Strata (thus not included in our baseline), using an SMT solver.

These validation experiments uncovered bugs~\cite{BugIntel,BugStoke983,BugStoke986} in the Intel manual, in Strata's simplification rules, and in the Stoke semantics.  All these bugs were reported to the respective authors, and most have been acknowledged and some have been fixed. Following is the brief summary of inconsistencies we found during the validation experiments.

\paragraph{Inconsistencies Found in the Intel Manual}
According to the manual, the semantics of \instr{vpsravd} \instr{\%xmm3,} \instr{\%xmm2,} \instr{\%xmm1} seems to depend on the lower $100$ bits of \reg{xmm3}, whereas the actual hardware execution suggests that it should depend on the lower $128$ bits. Similar inconsistencies are found in instructions with mnemonics \instr{vpsllvd}, \instr{vpsllvq}, \instr{vpsravd}.
Also, we found misleading typos related to instructions with opcodes \instr{vpsravw}, \instr{vpsravd}, \instr{vpsravq}, \instr{packsswb}.    
All these findings were reported and acknowledged by Intel as issues in the manual~\cite{BugIntel}.

\paragraph{Inconsistencies Found in \Strata's Simplification Rules}
While testing the instructions specifications borrowed from \Strata, we found inconsistent behaviors with the actual hardware. Moreover, the inconsistencies were discovered in the formulas of floating-point instructions. This is not surprising because \Strata models the floating-point instructions as \uif{} which cannot be executed or tested on hardware. Their semantics are executable in our definition though, and thus we were able to test them thoroughly. Note that Strata generates the formulas for these instructions by symbolically  executing  the corresponding learned  instruction sequences followed by a formula simplification pass. Therefore, errors in those formulas can be due to bugs either in the symbolic execution engine or in the simplification stage. Our testing shows that the second is true with the following evidence.
The simplification rule \s{add\_double(A, 0) == A} does not hold for \s{A = $-0.0$}. Same for \s{add\_single}. These were reported~\cite{PC1}.
Also, the simplification rule \s{sub\_double(A, A) == 0} does not hold for \s{A = $NaN$}.  Same is true for \s{sub\_single}. We found this bug in the  branch of Stoke which is used in \Strata. But this has been already fixed in the latest \Stoke branch.

\paragraph{Inconsistencies Found in \Stoke}

First, for instructions like \instr{addsubpd \%xmm1, \%xmm2} , the order of addition and subtraction specified by \Stoke is opposite to the one specified in the Intel Manual. Same is true with the mnemonic \instr{addsubps}. (Found in $12$ instruction variants.)

Second, the instruction \instr{pslld \%xmm1, \%xmm2} implements a logical left shift of packed data by a count specified in \reg{xmm1}. \Stoke's specification vectorized the operand \reg{xmm1} which is incorrect according to the manual. Similar issues were found in instructions implementing the logical right shift operations on packed data. (Found in  $18$ instructions.)

Third, \instr{cvtsi2sdl} \instr{\%eax,} \instr{\%xmm1} and \instr{vcvtsi2sdl} \instr{\%eax,} \instr{\%xmm0,} \instr{\%xmm1} are respectively SSE- and AVX-versions of the instruction to  convert a double-word ($32$-bit) integer to a scalar single-precision floating-point value. According to the manual, in the AVX-version,  the  destination bits $127-64$ of the  register \reg{xmm1} are updated to the corresponding bits in the first source operand \reg{xmm0}. This  is in contrast to the SSE-version of the instruction where the destination bits $127-64$ should remain unmodified. \Stoke specifies the semantics of the AVX-version similar to the SSE-version, which is incorrect.  (Found in $4$ instruction variants.)

Finally, some instructions, like \instr{imulb \%al}, which drive flag registers to an undefined state are not modeled correctly in \Stoke.    (found in $8$ instruction variants)

All these errors were reported and confirmed~\cite{BugStoke983,BugStoke986}.

\vspace{4pt}
Moreover, we illustrate the potential of our semantics to be used for formal analyses such as deductive program verification and program equivalence checking. For example, following are the three applications that we demonstrated:

\paragraph{Program Verification} To demonstrate that our semantics can be used to verify
x86-64 programs, we use the x86-64 verifier to prove the
functional correctness of the sum-to-n program. The functional correctness
property, which is $return-value = \sum_1^N n = N (N + 1) / {2}$ is specified as reachability specifications, essentially a pair of pre- and post-conditions for the function computing the sum-to-n.

\paragraph{Symbolic Execution to find Security Vulnerability}
\K automatically derives a correct-by-construction symbolic execution engine from the given semantics.
Being instantiated with our semantics, the engine can be used to symbolically execute and explore all possible paths in the given \ISA program.
We demonstrated how this capability can be used to find a  security vulnerability. We took a known security vulnerability using a code snippet of the HiStar~\cite{HiStar:2006} kernel. Using symbolic execution, we derive a path condition which can trigger the  vulnerability along with the triggering inputs.

\paragraph{Translation Validation of Optimizations}
\K also provides a program equivalence checker that can be used for the translation validation of compiler optimizations.
We derived an \ISA program equivalence checker from our semantics and used it to validate different optimizations of \s{popcnt} program which counts the number of set bits in the given input. We compiled the program with different optimizations: the GCC compiler optimizations (-O0, -O1, -O2, and -O3), and the Stoke super-optimization. We validated these optimizations by checking the equivalence between the optimized programs.


The \K framework also enables one to represent our semantics as SMT theories, which allows others to easily reuse our semantics for their own purposes.
\SC{%
    Indeed, we have translated our semantics to Stoke~\cite{completing-stock} which can serve as a drop-in replacement of Heule~\etal's semantics~\cite{Heule2016a} and can immediately benefit tools built on Stoke (e.g., \cite{Roessle:CPP19}).
}
%
% \paragraph{Artifacts}

Our formal semantics is publicly available at~\cite{x86-64-github}.


