\chapter{Future Work}\label{sec:problem}
\section{Problem Statement}\label{sec:statement}

We note that the faithfulness of the IR produced by the binary lifters, by
translating the binary, is pivotal to any binary analysis built on top of it.
Any bug in the translation would invalidate the binary analysis results. For
example, a malware analysis system might give false alarms or a binary
instrumentation system instrumenting a buggy IR might lead to failure or even
crash in interpreting the instrumented program. 

Given the importance of binary analysis, establishing the faithfulness of
binary lifter (or decompilers) is most desirable, which is what we propose as
our future work.

Having said that, we formulated our problem statement as follows:

\vspace{10pt}

\noindent\textbf{Problem Statement}: \emph{Developing binary-translator-agnostic tools and techniques to formally  
    validate the translation from binary to high level IR}

Specifically, we would like to validate the translation from binary to high
level IR considering the translator as a black-box, thereby, applying the
techniques to multiple state-of-art binary lifters/translators. This is
opposed to validating the translator\footnote{Translator validation uses a
  theorem prover or proof assistant to verify a translator correct, once and
    for all, so that all translator output is guaranteed correct for all valid
      source programs. Translation validation, on the other hand, runs a
        theorem prover after each translation to check that the output of the
        translator is semantically equivalent to the source program.}, as
         in  compiler verification~\cite{Leroy:2009}, which
        is not only more involving but is specific to the individual
        translators.

Next, we will discuss the recent efforts in validating the binary lifters in
Section~\ref{sec:recent-advances} and challenges in the proposed work (Section
    ~\ref{sec:challenges}) and the detailed steps (Section~\ref{sec:approach}) that we envision to
mitigate those.

\section{Recent Advances in Validation of Decompilation}\label{sec:recent-advances}

Despite of the importance in establishing the faithfulness of the binary
lifters, there has been surprisingly little effort towards that direction.
%
The most widely used approach in validating the binary lifters is based on
random testing, where the translator is considered as a black-box. Most notable
work along this line  is that of Martignoni et al.~\cite{Martignoni:ISSTA2009,
  Martignoni:ISSTA2010}, which proposes differential testing on
  QEMU~\cite{QEMU:USENIX05} and Bochs~\cite{Bochs1996}.  Specifically, they
  compared the state between a physical and an emulated CPU after executing
  randomly chosen instructions on both to discover any semantic deviations. In
  a related work, Martignoni \etal~\cite{Martignoni:ASPLOS2012} uses symbolic
  execution on a Hi-Fi emulator~\cite{Bochs1996}, defined as an binary emulator
  which is more complete in terms of instructions coverage of IA-32 ISA and
  faithful, to generate high-quality test cases to validate  a Lo-Fi
  emulator~\cite{QEMU:USENIX05}, defined as  less complete and buggier
  emulator. In both the above cases, validation works as follows: The binary
  and the lifted IR are executed independently, one on real hardware and the
  other on the emulator under test, and the output states are matched.
  Although, a simple and scalable approach, it's effectiveness is limited
  because many semantics bugs in binary lifters are triggered upon a specific
  input and exercising all such corner inputs is non-trivial. Note that, even
  though Martignoni \etal~\cite{Martignoni:ASPLOS2012} symbolically explored
  the test-cases which are supposed to covers all the paths of a given
  instruction's implementation, but being a differential testing-based
  approach, the faithfulness depends directly on  the faithfulness of the Hi-Fi
  emulator. A wrong implementation (or even omission of a particular case) of
  instruction semantics in the Hi-Fi, will lead to test-cases insufficient to
  explore all the paths and hence find bugs in the Low-Fi emulator. Moreover,
  their technique relies on the X86 the interpreter FuzzBALL. A bug in the
  later will effect the generation of high-fidelity test cases for a particular
  instruction, leading to incomplete coverage of the instruction's
  implementation in Low-Fi emulator. In summary, the test-based approach to
  validation the binary lifters are simple and scalable, but cannot be trusted
  to establish faithfulness.     

Another direction to establish guarantees in the binary translation is by using
formal methods. The most notable effort along the direction is the system
MeanDiff, developed by Soomin \etal~\cite{ASE2017}.
%
MeanDiff proposed a N-version IR testing to test three binary lifters,
         BAP~\cite{BAP:CAV11}, BINSEC~\cite{BINSEC2011}, and PyVEX~\cite{PYVEX}
         translating a single binary instruction  to BIL, DBA, and VEX IRs
         respectively). The tools symbolically execute each of the IR
         instances, lifted from a single binary instruction, to generate
         symbolic summaries to be compared using a SAT solver. MeanDiff neither
         handle floating point operations, nor the instructions which does not
         manifest their side-effects (like flag updates) explicitly. Moreover,
         MeanDiff reports a bug whenever a deviation is detected w.r.t the
         instruction-semantics-behavior in at least two binary lifters. But
         even if all the binary lifters are in sync on the behavior of a
         particular instruction, we cannot guarantee that all the lifters are
         faithful in lifting that instruction, which is however, a general
         limitation of differential testing based approach. Also, as MeanDiff
         is testing multiple binary lifters  together, hence it cannot be used
         to establish the faithfulness in lifting the semantics of an
         instruction which is not implemented in one of the lifters. 
%

Myreen et al.~\cite{Myreen:FMCAD:2008,Myreen:FMCAD:2012} proposed
proof-producing decompilation where they translates a sequence of machine code
to a tail-recursive functions defined in the language of a theorem prover,
   which accurately describes the effect of the given machine code and hides
   irrelevant details of the underlying machine language specification. Along
   similar lines,  a recent work by Roessle et al.~\cite{Roessle:CPP19}
   presents a method to extract the symbolic summary of a sequential binary
   program using the symbolic specification of instruction semantics extracted
   mostly from Strata. To the best of our knowledge, none of them has been used
   to validate the translation achieved by the state-of-the-art decompilers. 
%

Another recent work by Schulte \etal~\cite{eschulte2018bed} proposed
Byte-Equivalent Decompilation (BED) which leveraged a genetic optimization
algorithm to infer C source code from a binary. Given a target binary and an
initial population with decompilation candidates, they  drive a genetic
algorithm to improve the initial candidates, driving them closer to
byte-equivalence w.r.t the target binary. The byte equivalence  is simply the
edit distance to the target binary. As hypothesized in the future work section
of the paper~\cite{eschulte2018bed}, BED could be applied to LLVM IR instead of
C to evolve lifting from machine code to LLVM IR and may work well due to the
relative simplicity of LLVM IR as compared to the C. Being byte-equivalent the
generated LLVM IR will be the faithful evolution from the machine code.
However, as shown in the paper, this approach worked moderately well for
smallish binaries, it is still a challenge to realized a end-end binary to LLVM
decompiler which is correct by construction.

\section{Challenges in Translation Validation of Decompilers}\label{sec:challenges}
According to the pioneering work by Amir \etal~\cite{Pnueli:1998}, following
are three key ingredients to set up a fully automatic translation validation
system:

\subsection{Providing Common Semantic framework} This ingredient requires a
common semantic framework in order to represent both the source (as \ISA binary
    program) and the target code(as LLVM IR). Such a common framework allows
the source and target programs to be both represented using a common
specification language which assist reasoning over them. Moreover, the
semantics framework need to be general enough to model programs written in
different languages.

One of the long standing challenge, which deters many previous efforts like
~\cite{ASE2017}, in employing a formal validation approach in validating
binary lifters, is the lack of formal specification of the \ISA instructions
specifying binary program. Such instruction are massive in both numbers and
complexity and are specified in an informal manner in the Intel
manual~\cite{IntelManual} worth over $3,800$ pages. The \ISA ISA has a large
number of instructions, partly because of a large number of complex
instructions and partly because it keeps most of the legacy and deprecated
instructions ($~336$+) for the sake of backwards compatibility. It consists of
$996$ mnemonics, and each mnemonic admits several variants, depending on the
types (i.e., register, memory, or constant) and the size (i.e., the bit-width)
  of operands. Moreover, the x86-64 reference manual informally explains the
  instruction behaviors, leaving certain details unspecified or ambiguous,
  which requires  to consult with an actual processor implementation to clarify
  such details. Completely formalizing the vast number of instructions with
  carefully identifying all the corner cases from the informal document, thus,
  is highly non-trivial.

\subsection{Providing Formal Notion of Program Equivalence} This ingredient
expects the existence of designated control location(s) in the computations
of the source and target programs, that can serve as an observation point(s)
  for comparing the values of a set of externally observable variables
  (input/output variables, for example). The challenge here is to formally
  establish the notion that the lifted LLVM IR  correctly implements the \ISA
  binary code. Given the recent advances in translation validation in
  validating compilation, a basic version of this strategy is likely quite
  feasible today, however, there are additional challenges to deal with when it
  comes to validating the decompilation pipeline, like finding function, basic
  block and variable correspondence for checking equivalence, identifying and
  pruning glue code added during decompilation, to make the equivalence checker
  aware of the optimizations performed by the decompiler (e.g.  recovering
      function prototype, type etc).

\subsection{Providing automated proof generator and checker} This ingredient is about
 generating a machine-checkable proof  of valid translation for a given pair
of input and output programs as per the notion of equivalence established
above. The proof can be generated by  the binary-translator itself.
Intuitively, the proofs will try to establish equivalence between
synchronization points which are symbolic descriptions that capture the set of
pairs of relevant states of the input and output programs. However, we want 
the proof generator to be agnostic of the binary-translator involved, which 
makes the problem  even more challenging.   


\section{Proposed Approach}\label{sec:approach} In this section we would
outline our setup to validate the translation from binary to LLVM IR and
propose solution to mitigate the challenges mentioned in
Section~\ref{sec:challenges}. 


As the common semantic framework for the representation of the binary code and
the lifted LLVM IR code, we decided to build on K~\cite{k-primer-2013-v32},
    which is a one of the state-of-the-art frameworks for defining formal
    language semantics (Refer to Section~\ref{sec:related-work} for the other
        framework options). Given a syntax and a semantics of a language, K
        generates a parser, an interpreter, as well as formal analysis tools
        such as model checkers and deductive program verifiers, at no
        additional cost. Using the interpreter, one can test their semantics
        immediately, which significantly increases the efficiency of semantics
        developments. Furthermore, the formal analysis tools facilitate formal
        reasoning about the given language semantics. This helps in terms of
        both applicability of the semantics and engineering the semantics
        itself.
    
In formal methods literature, the notion of program (or semantic) equivalence
is usually formalized as a bi-simulation relation~\cite{Sangiorgi:2011} between
pairs of states of the two programs that are subject to prove equivalence
(which in turn are represented as state transition systems). Ideally, we want a
bi-simulation variant that allows us to simply relate the states where the two
programs actually synchronize, i.e., at the start of corresponding functions or
basic blocks, at the loop headers, etc. We borrowed the novel concept of 
bisimulation, called \emph{cut-bisimulation}, from ~\cite{}.
Cut-bisimulation is well-suited for translation  across language changes, as it
can ignore program points of no interest in the input and/or output
program.\Comment{Brief the concept}

As for the last ingredient of a \TV system, we would like to design them with
the two primary goals in mind: (1) Language independence, and (2) Translator
independence. Language independence is desirable so as to use the proof
generator and the checker to be used for   many different lifted IRs (like
    LLVM, BIL etc.). Indeed, the proof generator has to take into account the specifics
of the translation and some of the naive approaches to do that are by using
translator-generated hints or heuristics inspired by the translation's effect.
Both of these approaches  are translator specific and ad-hoc. Whereas, we would
like the proof generator to work uniformly across translators.   
 
 In order to achieve language-independence, we borrowed a
 language-independent equivalence checking algorithm that can be parameterized
 with the input and output language semantics. The algorithm employs the notion
 of cut-bisimulation, mentioned above, and the program point pairs needed for
 the proof are provided as an input to the algorithm, and are called
 synchronization points. Intuitively, synchronization points are symbolic
 descriptions that capture the set of pairs of relevant (concrete) states of
 the input and output programs.
 
 The above equivalence checking algorithm is implemented using the K
 framework~\cite{k-primer-2013-v32}, in a tool we call Keq that accepts formal
 semantic definitions for the input and output languages. In order to use Keq
 for \TV of the binary decompilation to LLVM IR, we have developed \K semantic
 definitions of the most complete user-level \ISA (Refer to Section
     ~\ref{sec:results}). Moreover, the semantics of a subset of LLVM IR is
 already available~\cite{LLVMSEMA} for us to leverage and build on. 
 
 \subsection{Generating Synchronization Points}
