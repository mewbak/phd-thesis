\chapter{Background on Decompilers: Facilitating Binary Analysis}\label{sec:decompilers}

Binary analysis is not easy~\cite{Meng:2016} and few long standing
challenges can be enumerated as follows:
\begin{itemize}
    \item Code and Data Ambiguity
    \item No Fixed Procedure Layout
    \item Missing or Untrusted Symbol Information
    \item Complex instruction Set
    \item Indirect Branches
    \item Overlapping Instructions
    \item Abusing Calls and Returns
    \item Lack of Types
    \item Presence of Non-returning functions
\end{itemize}

Despite the various challenges in analyzing machine code, there has been
impressive amount of work to rebuild a close approximation of the high-level
source code from a compiled binary using various decompilation
frameworks~\cite{McSema:Recon14,Remill,Angr1,BAP:CAV11,Radare2,FCD,BitBlaze:2008,hexray,Fokin:2011,eschulte2018bed,katz2018rnn,Schwartz:2013,IDA,mctoll,revgen}.

Binary analysis using a decompilation framework is achieved by (1) Binary
decompilation/lifting: Translating machine code to a intermediate
representation (IR), which precisely represents the operational semantics of
the binary code. The lifted IR exposes many high-level properties (like control
    flow, function boundary and prototype, variable and their type etc.) of the
the binary, which are otherwise lost during the compilation pipeline, and (2)
  performing the analysis at the IR level.  Analyzing the binary using the
  abstractions lifted to such high-level IR assist further analysis and/or
  optimization. We note that the IR, being the basis for any binary analysis
  techniques, the faithfulness of the lifting or decompilation process is
  highly desirable. 

Binary analyzes are mostly agnostic to any specific high-level IR, but many
projects~\cite{McSema:Recon14,Remill,FCD,reopt,mctoll} prefer to employ LLVM
IR~\cite{Lattner:2004}. LLVM IR, being an industry standard compiler IR,
  enables many analyses and optimizations out-of-the-box which allows building
  a static binary analyzer with minimal effort.

\Comment{Explain the output of some of the Decompilers and discuss the lifting choices they make}
